\documentclass{beamer}


\usepackage[frenchb]{babel}
\usepackage[T1]{fontenc}
\usepackage[utf8]{inputenc}
\usepackage{graphicx} 


\usetheme{CambridgeUS}

%\includegraphics{(nom de l'image)} pour insérer image .

\title[Projet chromosome]{Rétine}
\subtitle{L'évolution de la couleur des yeux .}
\author{Abdallah,Lallis}
\date{08/03/17}

\AtBeginSection[]
{
  \begin{frame}
  \frametitle{Sommaire}
  \tableofcontents[currentsection, hideothersubsections]
  \end{frame} 
}

\begin{document}

\begin{frame}
\titlepage
\end{frame}

\section{Introduction}
\subsection{Contexte}

\begin{frame}
\centering
\includegraphics[scale=0.405]{Population.jpg} 
\end{frame}

\subsection{Problèmatique}

\begin{frame}
\begin{block}{} 
Comment évoluent les proportions des couleurs des yeux au fil des générations dans une population?
\end{block}
\centering
\includegraphics[scale=0.405]{Chat_bicolore.jpg}  % Fixe l'échelle
\end{frame}

\subsection{Questions scientifiques}

\begin{frame}
\begin{block}{} 
\begin{itemize}
\item Comment évolue le taux de chaque couleur au fil des générations? \newline
\item Comment le taux de personnes au yeux bleu a t-il pu croître ?  \newline
\item Est-ce qu'un gène peut disparaître ? 
\end{itemize}
\end{block}
\end{frame}

\section{Modélisation}
\subsection{Préambule : l'hérédité de la couleur des yeux}

\begin{frame}
\includegraphics[scale=0.6]{AnnexeB.png}
\end{frame}


\begin{frame}
\includegraphics[scale=0.405]{AnnexeA.png}
\end{frame}


\subsection{Hypothèses}


\begin{frame}
\begin{block}{} 
\begin{enumerate}
\item Si M > B alors \%M > \%B
\item Le déplacement de population peut changer les proportions des couleurs. 
\item L'homogamie permet d'augmenter le taux d'une couleur à caractère récessif.
\item Un gène peut disparaître si sa proportion devient trop faible.
\end{enumerate}
\end{block}
\end{frame}


\subsection{Contexte de la modélisation}
\begin{frame}
\begin{block}{} 
\begin{itemize}
\item Modèle unisexe
\item Nombre d'enfants aléatoire ou fixe
\item Création d'échantillon précis ou global
\item Homogamie 
\end{itemize}
\end{block}
\end{frame}


\subsection{Description du modèle}

\begin{frame}
\centering
\includegraphics[scale=0.25]{AnnexeF.png}  % screen netlogo
\end{frame}


\section{Simulation}
\subsection{Simulation\&Résultat}


\begin{frame}
%\begin{block}{} 
%Sur un échantillon de 10000 individu , nous avons pris 10\% d'individu aux yeux bleu , 30\% %d'individu au yeux vert , 60\% d'individu aux yeux marron . Puis nous avons fait évoluer %l'échantillon sur 30 générations sans facteurs apparent.
%\end{block}
\centering
\includegraphics[scale=0.5]{hyp1.png}
\end{frame}


\begin{frame}
%\begin{block}{} 
%Sur un échantillon de 10000 individu , nous avons pris deux population distincte et les avons réuni %pour simuler un déplacement de personne.
%\end{block}
\centering
\includegraphics[scale=0.6]{hyp2.png}
\end{frame}

\begin{frame}
%\begin{block}{} 
%Sur un échantillon de 10000 individu , Nous pris 1\% de personne au yeux bleu et 99\% de personne au %yeux marron.Puis nous avons simuler 10 générations sans homogamie , 50 générations avec homogamie %deux fois d'affiler .
%\end{block}
\centering
\includegraphics[scale=0.45]{hyp3.png}
\end{frame}



\subsection{Conclusion}

\begin{frame}
\begin{block}{} 
\begin{itemize}
\item Les proportions de couleurs ne varient pas au fil des générations sans l'influence d'autres facteurs.
\item Le déplacement de population fait varier les taux en faveur des gènes dominants. 
\item L'homogamie permet à un gène récessif de se développer plus vite.
\end{itemize}
\end{block}
\end{frame}




\end{document}
